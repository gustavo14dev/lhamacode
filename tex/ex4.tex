\documentclass[10pt, aspectratio=169]{beamer}

% --- Pacotes de Design e Tipografia ---
\usepackage[utf8]{inputenc}
\usepackage[T1]{fontenc}
\usepackage{tikz}
\usetikzlibrary{shapes.geometric, shadows}
\usepackage{anyfontsize}
\usepackage{tcolorbox}

% --- Paleta de Cores "Vanguardista" ---
\definecolor{BgPaper}{HTML}{F5F5F0}
\definecolor{BauhausRed}{HTML}{E63946}
\definecolor{DeepSlate}{HTML}{1D3557}
\definecolor{GoldAccent}{HTML}{A89672}

% --- Configurações Básicas ---
\setbeamercolor{background canvas}{bg=BgPaper}
\setbeamercolor{normal text}{fg=DeepSlate}
\setbeamertemplate{navigation symbols}{}

% --- Template de Título ---
\setbeamertemplate{frametitle}{
    \vspace{0.4cm}
    \begin{tikzpicture}[remember picture, overlay]
        \fill[GoldAccent, opacity=0.2] (2, 0.5) circle (1.2cm);
        \node[anchor=west] at (0.5, 0) {
            \fontsize{22}{26}\selectfont \textbf{\textcolor{DeepSlate}{\insertframetitle}}
        };
        \fill[BauhausRed] (0.5, -0.6) rectangle (2.5, -0.7);
    \end{tikzpicture}
    \vspace{0.8cm}
}

% --- Capa ---
\setbeamertemplate{title page}{
    \begin{tikzpicture}[remember picture, overlay]
        \fill[DeepSlate] ([xshift=-6cm]current page.north east) rectangle (current page.south east);
        \fill[BauhausRed] ([xshift=-6cm, yshift=2cm]current page.center) circle (1.5cm);
        \node[anchor=west, text width=7cm] at ([xshift=1.5cm, yshift=1cm]current page.west) {
            {\fontsize{35}{40}\selectfont \textbf{\inserttitle}} \\[0.5cm]
            {\large \textcolor{GoldAccent}{\insertsubtitle}}
        };
        \node[anchor=west, text white] at ([xshift=-5.5cm, yshift=-3cm]current page.north east) {
            \begin{minipage}{5cm}
                \textcolor{white}{\textbf{\insertauthor}} \\
                \textcolor{BgPaper!70}{\footnotesize \insertinstitute} \\
                \vspace{0.2cm}
                \tikz \draw[white, thin] (0,0) -- (2,0);
            \end{minipage}
        };
    \end{tikzpicture}
}

\begin{document}

\title{DISRUPÇÃO}
\subtitle{Design Além do Óbvio}
\author{Seu Nome Aqui}
\institute{Laboratório de Inovação}

% Slide 1 - Capa
\begin{frame}[plain]
  \titlepage
\end{frame}

% Slide 2
\begin{frame}{O Conceito Bauhaus}
\begin{columns}
\column{0.5\textwidth}
\textbf{Menos é Mais?} \\
Não. O design aqui fala sobre \textbf{controle visual e hierarquia}.  
Cada elemento possui peso, função e silêncio.

\vspace{0.4cm}
Nada está aqui por acaso.
\column{0.5\textwidth}
\begin{tcolorbox}[colback=BauhausRed, colframe=BauhausRed, arc=0pt, boxrule=0pt]
\textcolor{white}{Design não é estética. É decisão visual.}
\end{tcolorbox}
\end{columns}
\end{frame}

% Slide 3
\begin{frame}{Forma Antes da Função}
O olhar humano responde primeiro à forma, depois ao conteúdo.

\vspace{0.4cm}
\begin{itemize}
\item[\color{BauhausRed}\tiny$\blacksquare$] Geometria orienta leitura
\item[\color{BauhausRed}\tiny$\blacksquare$] Cores criam tensão
\item[\color{BauhausRed}\tiny$\blacksquare$] Espaço comunica intenção
\end{itemize}
\end{frame}

% Slide 4
\begin{frame}{Design é Ritmo}
Texto, vazio, forma.  
Texto, vazio, forma.

\vspace{0.6cm}
\textit{O ritmo visual define a experiência antes da mensagem.}
\end{frame}

% Slide 5
\begin{frame}{Equações como Arte}
\vspace{1cm}
\begin{center}
{\fontsize{30}{35}\selectfont
\[
\oint_C \mathbf{B} \cdot d\mathbf{l} = \mu_0 I_{enc}
\]
}
\vspace{0.5cm}
\tikz \draw[GoldAccent, line width=3pt] (0,0) -- (1,0);\\
\small Informação também pode ser bela.
\end{center}
\end{frame}

% Slide 6
\begin{frame}{Tipografia como Arquitetura}
Cada fonte cria uma estrutura invisível.

\vspace{0.4cm}
\begin{itemize}
\item Peso define autoridade
\item Espaçamento cria conforto
\item Tamanho define prioridade
\end{itemize}
\end{frame}

% Slide 7
\begin{frame}{Cores Não Decoram}
Cores \textbf{direcionam decisões}.

\vspace{0.5cm}
\begin{tcolorbox}[colback=GoldAccent!30, colframe=GoldAccent, arc=0pt, boxrule=0pt]
Contraste é narrativa visual.
\end{tcolorbox}
\end{frame}

% Slide 8
\begin{frame}{Assimetria Controlada}
A quebra do padrão chama atenção.

\vspace{0.4cm}
Assimetria gera dinamismo sem caos — quando bem calculada.
\end{frame}

% Slide 9
\begin{frame}{Design é Estratégia}
Não existe design neutro.

\vspace{0.5cm}
Tudo influencia:
\begin{itemize}
\item Decisão
\item Emoção
\item Comportamento
\end{itemize}
\end{frame}

% Slide 10 - Encerramento
\begin{frame}[plain]
\begin{tikzpicture}[remember picture, overlay]
\fill[BauhausRed] (current page.south west) rectangle (current page.north east);
\node at (current page.center) {
\begin{minipage}{\textwidth}
\centering
\fontsize{50}{60}\selectfont \textbf{\textcolor{white}{OBRIGADO.}}\\
\vspace{0.5cm}
\large \textcolor{DeepSlate}{Vamos conversar?}
\end{minipage}
};
\end{tikzpicture}
\end{frame}

\end{document}
